%%%%%%%%%%%%%%%%%%%%%%%%%%%%%%%%%%%%%%%%%
% Journal Article
% LaTeX Template
% Version 1.3 (9/9/13)
%
% This template has been downloaded from:
% http://www.LaTeXTemplates.com
%
% Original author:
% Frits Wenneker (http://www.howtotex.com)
%
% License:
% CC BY-NC-SA 3.0 (http://creativecommons.org/licenses/by-nc-sa/3.0/)
%
%%%%%%%%%%%%%%%%%%%%%%%%%%%%%%%%%%%%%%%%%

%----------------------------------------------------------------------------------------
%	PACKAGES AND OTHER DOCUMENT CONFIGURATIONS
%----------------------------------------------------------------------------------------

\documentclass[twoside]{article}

\usepackage{lipsum} % Package to generate dummy text throughout this template
\usepackage{bm}
\usepackage{tabularx}
\usepackage{etoolbox}
\apptocmd\normalsize{%
 \abovedisplayskip=12pt plus 3pt minus 9pt
 \abovedisplayshortskip=0pt plus 3pt
 \belowdisplayskip=12pt plus 3pt minus 9pt
 \belowdisplayshortskip=7pt plus 3pt minus 4pt
}{}{}

\usepackage[sc]{mathpazo} % Use the Palatino font
\usepackage[T1]{fontenc} % Use 8-bit encoding that has 256 glyphs
\linespread{1.05} % Line spacing - Palatino needs more space between lines
\usepackage{microtype} % Slightly tweak font spacing for aesthetics
\usepackage{floatrow}

\usepackage[hmarginratio=1:1,top=32mm,columnsep=20pt, outer=1.5cm]{geometry} % Document margins
\usepackage{multicol} % Used for the two-column layout of the document
\usepackage[hang, small,labelfont=bf,up,textfont=it,up]{caption} % Custom captions under/above floats in tables or figures
\usepackage{booktabs} % Horizontal rules in tables
\usepackage{float} % Required for tables and figures in the multi-column environment - they need to be placed in specific locations with the [H] (e.g. \begin{table}[H])
\usepackage{hyperref} % For hyperlinks in the PDF
\usepackage{amsmath,amssymb}
\usepackage{graphicx}
\usepackage{lettrine} % The lettrine is the first enlarged letter at the beginning of the text
\usepackage{paralist} % Used for the compactitem environment which makes bullet points with less space between them
\usepackage{subcaption}
\usepackage{appendix}


\usepackage{abstract} % Allows abstract customization
\renewcommand{\abstractnamefont}{\normalfont\bfseries} % Set the "Abstract" text to bold
\renewcommand{\abstracttextfont}{\normalfont\small\itshape} % Set the abstract itself to small italic text

\usepackage{titlesec} % Allows customization of titles
\renewcommand\thesection{\Roman{section}} % Roman numerals for the sections
\renewcommand\thesubsection{\Roman{subsection}} % Roman numerals for subsections
\titleformat{\section}[block]{\large\scshape\centering}{\thesection.}{1em}{} % Change the look of the section titles
\titleformat{\subsection}[block]{\large}{\thesubsection.}{1em}{} % Change the look of the section titles
\newcommand{\bigO}[1]{\ensuremath{\mathop{}\mathopen{}\mathcal{O}\mathopen{}\left(#1\right)}}
\def\mean#1{\left< #1 \right>}

%----------------------------------------------------------------------------------------
%	TITLE SECTION
%----------------------------------------------------------------------------------------

\title{\vspace{-15mm}\fontsize{24pt}{10pt}\selectfont\textbf{Monte Carlo Simulation of 2D Ising Model}} % Article title

\author{
\large
\textsc{Andr\'e Melo 4519302}\\
\textsc{Matteo Domenighini 4512154} \\[2mm] % Your name
\normalsize Delft University of Technology\\ % Your institution
\vspace{-5mm}
}
\date{}

%----------------------------------------------------------------------------------------

\begin{document}

\maketitle % Insert title

%----------------------------------------------------------------------------------------
%	ABSTRACT
%----------------------------------------------------------------------------------------

\begin{abstract}

\noindent The aim of this report is to present the results obtained using Monte Carlo simulation for the Ising model. Two different algorithms have been implemented to simulate the behaviour of the system: the Metropolis Monte Carlo algorithm and the Hoshen-Kopelman cluster finding algorithm. Both algorithms have been used to extrapolate relevant physical quantities, such as the magnetization, the magnetic susceptibility and the specific heat. Finite-size scaling has also been used in order to calculate the critical exponents. For the Metropolis algorithm, we tried to find a proof of the critical exponent universality by considering second neighbour interaction. The algorithms have been compared: the Hoshen Kopelman revealed itself to be better suited for (...)% Dummy abstract text

\end{abstract}

%----------------------------------------------------------------------------------------
%	ARTICLE CONTENTS
%----------------------------------------------------------------------------------------

\begin{multicols}{2} % Two-column layout throughout the main article text

\section{Introduction}
The Monte Carlo algorithms are a type of algorithms in which "random" numbers play an essential role ~\cite{thijssen}. This method has been widely and successfully implemented in the past decades to simulate the behaviour of physical systems in order to extrapolate information regarding their static properties. \\
The Hoshen Kopelman algorithm is a cluster finding algorithm that has been implemented as a non-recursive alternative to the Swendsen-Wang algorithm. It is based on the more famous Union-Find algorithm.

In section II the interaction model and the working principles of the two algorithms are outlined. In section III the numerical results obtained for specific heat, magnetization, magnetic susceptibility, Binder cumulant and critical exponent are reported. In section IV, the accuracy and speed of the two algorithms are compared.

%------------------------------------------------

\section{Methods}
While implementing the models and algorithm outlined in this section, we have considered $k_B = 1$, $J = 1$ and a unitary distance between the lattice sites.

\subsection{Ising Model}
The Ising Model consister in a lattice of size $L \times L$ in which every lattice site is associated with a spin. The Hamiltonian of the system in absence of an external magnetic field is described by

\begin{equation}
H = - J \sum_{\mean{ij}} \textbf{s}_i \textbf{s}_j
\end{equation}

where $\textbf{s}_i$ represents the spin associated with the \emph{i}th site.
We consider $J > 0$, which means that the system favours the parallel alignment of adjacent spins.
The partition function for the system is given by 

\begin{equation}
\mathcal{Z} = \sum_{\textbf{s}_i} e^{-\beta H\left(\textbf{s}_i\right)}
\end{equation}

which is dependent on the temperature of the system, contained in $\beta$.

\subsection{Metropolis Monte Carlo Algorithm}
The Metropolis Monte Carlo algorithm is an algorithm in which the information regarding different distributions of a system is stored in a Markov chain. 

\subsection{Hoshen-Kopelman Algorithm}


%------------------------------------------------

\section{Thermodynamic quantities}

\subsection{Pair Correlation Function}

\subsection{Energy}

\subsection{Pressure}

\subsection{Specific Heat}

\subsection{Diffusion Coefficient}

%------------------------------------------------
\section{Conclusion}
lalila

\begin{appendices}
\section{Appendix - Data blocking}

\end{appendices}

%----------------------------------------------------------------------------------------
%	REFERENCE LIST
%----------------------------------------------------------------------------------------

\begin{thebibliography}{99} % Bibliography - this is intentionally simple in this template

\bibitem{thijssen}
J.M.Thijssen,
\newblock {\em Computational Physics}, Cambridge University Press, 2nd Edition, 2007.
\end{thebibliography}

%----------------------------------------------------------------------------------------
\end{multicols}

\end{document}
 
